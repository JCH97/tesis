\begin{spanish_abstract}
	La estadística tradicional se ocupa principalmente de la asociación entre variables pero no revela información acerca de las relaciones de causalidad entre estas. A partir de estas limitaciones es necesario la construcción de una teoría que formalice matemáticamente los procesos causales. Dicha teoría permitirá modelar situaciones de la vida cotidiana y responder preguntas causales acerca de estas. Un objetivo más ambicioso consiste en la simulación del razonamiento causal humano, el cual se cree que debe ser una pieza fundamental en la construcción de una inteligencia artificial fuerte. En el presente trabajo se presentarán los principales desarrollos recientes de teorías matemático computacionales de la causalidad, con especial énfasis en el trabajo de Judea Pearl, y se proveerá una implementación de un modelo causal capaz de responder distintos tipos de preguntas causales. Por último, se exponen posibles escenarios donde se puede utilizar el programa propuesto.
\end{spanish_abstract}

