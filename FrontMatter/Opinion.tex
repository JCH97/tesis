\begin{opinion}
	Durante los últimos meses José Carlos ha venido desarrollando la tesis de diploma “Sistema para la Gestión de Horarios”  que constituye uno de los requisitos para convertirse en licenciado en ciencias de la computación. El trabajo de José Carlos consiste en una herramienta capaz gestionar el proceso de confección de horarios de inicio a fin. Las personas que crean los horarios docentes, podrán crear la planificacion docente de la facultad a través de un sitio web cómodo y con varias facilidades. 

	
	Durante ese período Jose Carlos fue capaz de leer bibliografía actualizada, aprender tecnologías modernas, utilizar patrones de diseño y buenas prácticas de programación. Jose Carlos trabajó de manera muy independiente y en las mayoría de las situaciones fue capaz de resolver los problemas a los que se enfrentó con mucha destreza, lo que demuestra la solidez de los conocimientos adquiridos durante la carrera. Integró conocimientos y habilidades de diferentes asignaturas y disciplinas de varios años.

	
	Creemos que está listo para obtener su título de licenciado en ciencias de la computación y enfrentarse a la vida profesional como una persona muy capaz y con sobrados conocimientos técnicos.

	
	Por todo lo antes expuesto, considerando la actualidad del trabajo realizado y su gran utilidad, propongo la calificación de excelente.\\
	

	La Habana, 24 de noviembre de 2022.
%	
%	\includegraphics[width=100px, height=80px]{./images/opinion.png}

	Lic. Pedro Rubén Quintero Rojas
	
	Facultad de Matemática y Computación	
	
	Universidad de La Habana, Cuba
	
\end{opinion}