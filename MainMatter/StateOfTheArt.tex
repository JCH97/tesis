\chapter{Antecedentes}\label{chapter:state_of_the_art} 

El tema del manejo, creación y gestión de un sistema de horarios ha sido abordado por un gran número de personas, en muchos ámbitos diferentes.

Usualmente el proceso de confección de un horario para un centro universitario sigue una serie de puntos o pautas, dígase por ejemplo:
\begin{enumerate}
	\item Definir la cantidad de turnos que se impartirán en un día lectivo.
	\item Definir la disposición de las aulas activas para albergar los turnos de clase.
	\item Planificar los planes de estudio por cada uno de los años.
	\item Definir una distribución para cada una de las asignaturas que se involucren  dentro del punto anterior. En dicha distribución se debe evitar colisiones entre los participantes.
	\begin{enumerate}
		\item En un escenario ideal en dicha distribución se debería considerar quizá la opinión de los profesores para garantizar un sistema lo más ajustado posible a las necesidades de todas los involucrados en el proceso.
	\end{enumerate}
\end{enumerate}

Lo mejor sería contar con una aplicación que brindara el mayor soporte posible a la persona que se encarga de confeccionar el sistema; para así garantizar con el ello que el trabajo resultara lo mas sencillo posible y que durante la tarea de planificar la distribución se compruebe de forma automática las posibles colisiones entre cada uno de los turnos, locales, asignaturas y profesores.

\section{Herramientas de gestión}
El ejemplo más concreto que se analizó antes de la creación del presente trabajo fue un sistema desarrollado en 2019 entre un grupo de universidades de América del Norte y Europa: \href{https://www.unitime.org/}{UniTime}\cite{UniTime}. Este sistema ofrece soporte en un gran número de escenarios, dígase por ejemplo la creación de una especie de salas para la planificación de eventos así como el manejo de secciones individuales por estudiantes. Además cuenta con una pequña comunidad que brinda soporte al mismo por lo que se realizan versiones y modificaciones periódicamente.

Otra herramienta que también llamo la atención fue \href{http://www.educaria.es/#horarios2}{Unit}\cite{Unit}. Este software confecciona automáticamente los horarios del colegio a partir de los criterios pedagógicos que se determinen en el centro. Propone al usuario una serie de alternativas diferentes sobre las que se pueden hacer cambios manuales. Incorpora además otras funcionalidades como la posibilidad de crear horas de entrada y recreos de los alumnos a horarios diferentes, rangos de horas en que los profesores deben impartir clases por contrato, horario óptimo de guardias, planificación de sustituciones o estadísticas, entre otras. Este generador de horarios está completamente integrado en la plataforma de gestión Alexia.\cite{Alexia}

Otro sistema que resaltó entre los revisados fue \href{https://www.penalara.com/es/CU}{Peñalara GHC}.\cite{Penalara_GHC} Se trata de una aplicación que permite llevar a cabo la gestión completa de los horarios escolares teniendo en cuenta los requisitos académicos, pedagógicos y organizativos del centro. Se muestra como buen candidato para todo tipo de niveles e instituciones educativas, pues organiza de forma objetiva y mediante la elección de un perfil de usuario los horarios semanales. Tiene la capacidad de resolver cualquier problema de planificación de profesores, espacios, sesiones lectivas. Además, es posible acceder a los archivos del software a través de Internet así como exportar los horarios a gestores de calendarios.

La principal cuestión que motivó el desarrollo de un sistema propio de la universidad y la no utilización de estas herramientas que ya existían fue la posibilidad de ofrecer un manejo de restricciones sobre todas las entidades del horario; restricciones que ofrecen un grado de \emph{felicidad} al ser evaluadas y permiten la apreciación por parte del creador del horario de que tan bueno resulta la distribución brindada a los turnos de clases; que viene siendo, en definitiva, el punto central de todas estas herramientas. 

Otros autores manejan en sus sistemas un concepto similar al de \emph{restricción}, pero estas están en todos los casos relacionadas con un profesor y un turno de clase; en cambio en el presente software se permite asociarlas a cualquier entidad definida dentro del sistema, dígase por ejemplo: \emph{local}, \emph{departamento}.

\section{Restricciones}
\label{state_of_art:restrictions}
El uso de restriccioens es una de las partes más llamativas de este sistema, la idea detrás del manejo e implementación de las mismas surge a través del trabajo de diploma desarrollado por el estudiante \textit{Joel Rey Travieso Sosa} \cite{thesis_joel} perteneciente a la Facultad de Matemática y Computación de La Universidad de La Habana.

El análisis y la gestión de las restricciones trajo consigo la clasificación en diferentes grupos:
\begin{itemize}
	\item \textit{Asignación de recursos}: Cuando un recurso debe ser asignado a otro recurso de distinto tipo o a cierto evento
	\item \textit{Asignación de tiempo}: Cuando un evento o recurso debe asignarse a una fecha
	\item \textit{Restricciones de tiempo entre eventos}: Cuando un evento mantiene una relación de tiempo con otro
	\item \textit{Solapamiento de eventos}: Cuando un grupo de eventos comparte un mismo intervalo de tiempo.
	\item \textit{Coherencia entre eventos}: Intentar producir horarios más organizados y convenientes
	\item \textit{Capacidad}: Algunos de los recursos tienen capacidades de uso que no pueden ser vulneradas.
	\item \textit{Continuidad}: Cuando se intenta producir horarios con determinadas características constantes o muy predecibles.
	
\end{itemize}

Además se detectaron cinco grupos fundamentales para clasificar las condiciones o requerimientos del problema:
\begin{itemize}
	\item \textit{Restricciones unarias}: aquellas que involucran un sólo evento, como por ejemplo, las clases de un curso no pueden ser programadas un día lunes.
	\item \textit{Restricciones binarias}: aquellas que involucran dos eventos. Un ejemplo típico son las restricciones de topes de horarios para un curso que requiere un mismo recurso: profesor, sala de clases, etc
	\item \textit{Restricciones de capacidad}: las que se imponen al asignar cursos a salas de clase con capacidad suficiente.
	\item \textit{Restricciones de separación de eventos}:  aquellas que requieren que las actividades estén separadas o siguiendo algún patrón en el tiempo. Algunos ejemplos son las impuestas por políticas de la institución de respetar asignaciones de horarios en patrones predefinidos o las condiciones de no existencia de horas intermedias vacías.
	\item \textit{Restricciones asociadas a los  agentes}: las limitaciones en los horarios asignados para cumplir con las preferencias de los profesores.
\end{itemize}

En la implementación que se ofrece se manejan 4 tipos de restricciones fundamentales, que cubren la descripción antes mencionada.
\begin{itemize}
	\item Restricción de requerimiento de cuenta simple.
	\item Restricción de requerimiento de cuenta de condiciones.
	\item Restricción de requerimiento de distribución de atributos.
	\item Restricción de requerimiento relacional.
\end{itemize}
En los capítulos siguientes se ofrecerá una explicación más detallada de todo lo relacionado con estas condiciones impuestas sobre el sistema así como la adecuada definición y formulación de todos los conceptos antes expuestos.