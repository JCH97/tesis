\begin{conclusions}
	Se realizó un recorrido general por la historia del desarrollo matemático y filosófico de la causalidad. El problema recayó primeramente en manos de los filósofos y comenzó a despertar el interés de algunos matemáticos a principios de la época moderna. Sin embargo fue abandonado tempranamente dado que estos desviaron su atención hacia la estadística, área de la matemática que también daba sus primeros pasos por aquel entonces. La evasión de la causalidad por parte de los estadísticos condujo al surgimiento de numerosas paradojas y problemas intratables por la falta de las herramientas adecuadas para encararlos, herramientas que no podía proporcionar la visión frecuentista de la estadística. La historia de la concepción matemática de la causalidad tiene su punto de inflexión con la publicación de Sewall Wright de los diagramas de caminos. Posteriormente comenzaron a desarrollarse diversas teorías y modelos que explican la causalidad, con Reichenbach, Suppes, Granger y Pearl entre sus principales exponentes.
	
	Se expusieron los desarrollos de Judea Pearl en la teoría de la causalidad y los modelos gráficos probabilistas. Se repasó la teoría de la independencia condicional como base para el desarrollo de los modelos gráficos probabilistas. Se presentaron las redes bayesianas como un primer modelo capaz de realizar predicciones en las variables a partir de observaciones. Posteriormente se introdujo el modelo causal estructural de Pearl como modelo gráfico probabilista capaz de representar explícitamente las relaciones de causalidad entre variables y su relación con las redes bayesianas. A continuación se pasó a introducir las distintas modalidades de inferencia causal propias de un SCM. Primeramente se formalizó el concepto de intervención, definiendo el operador \textit{do} y resaltando sus diferencias con la observación pasiva. A continuación se vieron los principales métodos de inferencia causal para el cálculo de intervenciones: el criterio de la puerta trasera, el criterio de la puerta principal, el cálculo-do y la inferencia bayesiana. Luego se expuso la teoría relacionada con los contrafactuales, sus variantes determinista y no determinista y los principales métodos para calcularlos, haciendo especial hincapié en el método de las redes gemelas. Por último se expusieron dos aplicaciones de las intervenciones y los contrafactuales conocidos como atribución y mediación. La primera permite explicar el comportamiento de una variable a partir del de otras y la segunda, el análisis del efecto que ejerce una variable sobre otra a través de un mediador.
	
	Se desarrolló una implementación de los algoritmos de inferencia causal, utilizando la idea de la cirugía en el grafo para intervenir el modelo y el método de las redes gemelas para calcular los contrafactuales, conjuntamente con la idea de transformar el modelo a una red bayesiana para aplicar inferencia bayesiana y obtener la respuesta a la consulta deseada.
	
	Los algoritmos de inferencia causal se comportaron satisfactoriamente para modelos cuyos grafos son esparcidos y las variables toman una cantidad relativamente pequeña de valores. Sin embargo, cuando el número de nodos, estados y aristas del grafo causal crece notablemente, el tiempo de ejecución se ve comprometido, especialmente en el caso de los contrafactuales, en los cuales la inferencia es aplicada en un grafo que en el caso peor ocupa el doble de tamaño que el grafo original. La técnica de mezcla de nodos resultó ser útil para disminuir el costo computacional de calcular los contrafactuales.
	
	Además se desarrolló una interfaz visual destinada a usuarios no expertos en programación para la resolución de problemas de naturaleza causal.
	
	Por último se mostraron algunas aplicaciones de la teoría de la causalidad en las que puede ser utilizado el programa implementado.
	
	La causalidad ha demostrado ser una herramienta útil para complementar el procesamiento estadístico de los datos. Su uso permite la modelación y el entendimiento de procesos que la asociación no captura. Simular intervenciones a partir de datos recopilados, en vez de llevar a cabo la intervención en la práctica, permite no solo el ahorro de recursos, sino también conocer el resultado de intervenciones que serían imposibles o inviables de determinar experimentalmente.
	
	Para concluir, se debe resaltar que la simulación del pensamiento causal nos colocará un paso más cerca de simular el pensamiento humano. Dotar de razonamiento causal a las máquinas permitirá que estas adquieran capacidades distintivas del ser humano, como son planear acciones con antelación o aprender de los propios errores. Conforme surjan nuevas teorías que mejoren, extiendan y complementen las actuales y se desarrollen modelos más completos y realistas capaces de razonar casualmente, seremos capaces no solo de automatizar tareas complejas, sino además de entendernos mejor a nosotros mismos.	
\end{conclusions}