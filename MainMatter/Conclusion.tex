\chapter{Conclusiones}

%\begin{conclusions}
\section{Breve resumen del trabajo}

En una primera aproximación a la descripción de la implementación ofrecida, se intenta brinadar una especie de marco histórico acerca del tema que nos envuelve; haciendo alusión en varias ocasiones a las motivaciones que llevaron al desarrollo de un software propio, para gestionar el horario dentro de la Facultad de Matemática y Computación de la Universidad de La Habana. Dichos aspectos motivacionales están claramente relacionados con las ventajas que traería la puesta en práctica de una aplicación con tales fines. 

Los objetivos que se persiguen se encuentran descritos además, en las primeras secciones del documento, ya que resulta de notable importancia tenerlos claros para lograr obtener un resultado que cuente con la mayor calidad posible y que logre cubrir todos las necesidades y carencias de la facultad en este aspecto de la gestión del horario. 

El tema de la gestión de horarios para instituciones educacionales ha sido abordado por diversos autores; previo a la confección del software se realizó el análisis de las aplicaciones de mayor relevancia. En el capítulo 2 se exponen las conclusiones a las que se arribó luego del análisis, lo que contribuyó además a mostrar la utilidad de una aplicación propia que reflejara verdaderamente los intereses de la facultad.

Uno de los temas más novedosas o de mayor relevancia fue el manejo de restricciones, así como la incoprarción del concepto de felicidad del sistema a través del cumplimiento (o no) de estas. Una restricción, conceptualmente, se compone de un conjunto de condiciones y otro de requerimientos.  Las restricciones no otorgan felicidad por sí solas a ninguna entidad del sistema, son solamente una plataforma para representar la relación entre condiciones y requerimientos.
Una condición se compone por tres elementos: un atributo \textit{A} en primer lugar; y en función de este, un operador \textit{O} y un valor \textit{V}. A partir de varias condiciones es posible definir además, grupos de condiciones. 

Descripciones de todos los aspectos relacionados con los detalles de la implementación ofrecida y el enfoque de solución son abordados también en el documento.

\section{Cumplimiento de objetivos}

Se considera que los objetivos expuestos en la primera sección del trabajo en cuestión fueron cumplidos. A fecha de la entrega de este documento se cuenta con una aplicación web, desarrollada en NestJS(backend) y VueJS(frontend) que satisface todas las necesidades y carencias de la Facultad de Matemática y Computación con respecto al uso del horario. 

La aplicación se muestra lo suficientemente robusta como para albergar el manejo de todos los profesores, locales, grupos, asignaturas y demás aspectos relacionados con el funcionamiento interno de un centro educacional. Es capaz de identificar problemas de colisiones entre turnos, profesores y locales; pudiendo reaccionar en consecuencia y ofreciendo una respuesta visual al usuario. Así mismo cuenta con una serie de vistas que hacen posible la fácil interacción y que la experiencia de usuario, por tanto, se mejore considerablemente. Se integraron también una serie de filtros para garantizar que se pudiera visualizar en cualquier momento secciones específicas del horario; lo que proporciona, como es fácil notar, muchas bondades en el trabajo con el sistema, tanto para la persona encargada de crear el horario, como para el cliente que solo se propone consultar el mismo. 

La generación de reportes en formato Excel, también se cubrió de manera satisdactoria. Es posible obtener en todo momento un archivo que se compone de una serie de páginas: una por cada grupo y una extra para la visualización de los locales. 

El manejo de las condiciones de los profesores sobre el sistema, también se considera cumplido. En todo momento, el administrador puede consultar cuáles son las restricciones que están afectando la felicidad, por su no cumplimiento y actuar en consecuencias, garantizando la mayor comodidad para todos los trabajadores del centro. Por otra parte, a los profesores se les dió la potestad de definir, por ellos mismos, las restricciones y evitar con esto la necesidad de que un administrador las tenga que incorporar personalmente. 

El sistema, es capaz de manejar la incorporación de varias facultades a la vez e incluso está diseñado para garantizar que se puedan manipular varias universidades al mismo tiempo, aunque es válido notar, que en la mayoría de los casos solo se va a interactuar con una. 

Para corroborar además el cumplimiento de estos aspectos, se realizó la creación de un horario real dentro del sistema; el horario del 2$^{do}$ semestre del curso 2021-2022 de la facultad de Matemática y Computación de La Universidad de La Habana, fue el elegido para esto. 

\section{Recomendaciones}

Usando como base el trabajo realizado, la experiencia previa acumulada y las opiniones de un grupo de usuarios acerca del tema que se discute se hace posible presentar una serie de recomendaciones o ideas que quizá puedan ser evaluadas en futuras tareas.

Una mejora sustancial que se podría aplicar a la implementación ofrecida pudiera ser, sin dudas, \textbf{la utilización de GraphQL} \cite{GraphQL_doc} como forma de comunicación entre frontend y backend. La inserción de este lenguaje de consultas podría contribuir a que se ganara en performance, pues el frontend solo consumiría de la API los datos exactos que necesite en todo momento además de ofreceer la posibilidad de solicitar recursos anidados en la misma operación. Esto se refleja inmediatamente en una optimización de la red, reduciendo las cargas útiles de HTTP y el número de peticiones. 

Una de las características que también resultaría de gran utilidad sería \textbf{la inclusión de tests} dentro de la aplicación. La realización de pruebas al software es considerado por muchos autores, como buenas prácticas a la hora del desarrollo y sin dudas las ventajas son alentadoras; por ejemplo: \cite{testing} 
\begin{itemize}
	\item La calidad de código mejora puesto que podemos detectar errores en una etapa más temprana de desarrollo y de forma más rápida.
	\item Puedes trabajar de una forma más ágil, ya que facilita los cambios y favorece la integración.
	\item Los propios test pueden funcionar como documentación y ejemplos.
	\item Reduce el costo de mantenimiento del proyecto.
	\item A través del desarrollo guiado por pruebas, TDD por sus siglas en inglés, se ofrece una mejora en el diseño del software.
	\item Es posible detectar errores, antes de la puesta en producción, de nuevos cambios o implementaciones; que en un gran número de casos afectan funcionalidades realizadas previamente.
\end{itemize}

Resulta notable también analizar la forma de \textbf{incluir una generación automática del sistema de horarios}, a partir de una previa definición de frecuencias por asignaturas y el manejo de restricciones, punto que se encuetra actualmente abordado dentro del sistema.

Esta temática de la generación automática, fue previamente desarrollada como tema de tesis dentro de la Facultad de Matemática y Computación de la Universidad de La Habana y ponerlo en práctica aseguraría que se eliminara por completo el tiempo empleado por la persona encargada de la creación del horario; haciendo que la tarea resultara en extremo sencilla y que solo hiciera falta, en el peor de los casos, la intervención humana para quizá brindar soporte a la distribución ofrecida. 


\textbf{La puesta en producción de una aplicación móvil} para la visualización del horario gozaría de una aceptación considerable. Este tipo de aplicación haría posible que se garantizara un acceso más fácil al sistema  y que se mejorara aún más la experencia de usuario. Las notificaciones push son un ejemplo de ello, ya que estas proporcionan la información que se necesita, en el momento que se requiera. 














%\end{conclusions}